\documentclass{beamer}
\usepackage[UTF8]{ctex}
\usetheme{PaloAlto}
%\useoutertheme{shadow}
\useinnertheme{circles}
\usecolortheme{albatross}
\usefonttheme{structuresmallcapsserif}
\definecolor{bottomcolor}{rgb}{0.32,0.3,0.38}
\definecolor{middlecolor}{rgb}{0.08,0.08,0.16}
\setbeamertemplate{background canvas}[vertical shading][bottom=bottomcolor,middle=middlecolor,top=black]
%\setbeamerfont{title}{size=\LARGE}
%\setbeamercolor{title}{fg=yellow,bg=gray}
%\setbeamertemplate{section in toc}[sections numbered]%节标题模板颜色
%\setbeamercolor{section in toc}{fg=yellow!80!gray}%节标题模板颜色
\setbeamertemplate{frametitle}{\noindent\insertframetitle\par\noindent\insertframesubtitle\par}
\setbeamerfont{frametitle}{size=\large}
%\setbeamercolor{frametitle}{fg=yellow!70!gray}
\setbeamercolor{normal text}{fg=white,bg=black}
\setbeamertemplate{blocks}[rounded][shadow=true]
\setbeamercolor{block title}{fg=white,bg=gray!50!black}
%\setbeamercolor{block body}{bg=gray}
\usepackage[T1]{fontenc}
\usepackage{amsmath}%常用的数学宏包
\usepackage{amssymb}%数学宏包,可以实现一些数学符号
\usepackage{graphicx}
\usepackage{float}
\RequirePackage{caption}
\usepackage{subfigure}
\usepackage{fancyhdr}
\usepackage{listings}
\usepackage{pxfonts}
\usepackage{geometry}
\usepackage{setspace}
\usepackage{amsfonts}%数学字体宏包
\usepackage{comment}%批量注释的宏包
\usepackage{boxedminipage}%为文本段添加边框的宏包
\usepackage{shadow}%带阴影的边框
\usepackage{fancybox}%实现边框效果的宏包
\usepackage{esint}%实现多重积分的宏包
\usepackage{relsize}%提供加大数学符号命令\mathlarger

\title{向量内积相关题目(第5题)}
\subtitle{数学的奥秘:本质与思维}
\author{Itachi酱}
\date{2022年5月9日}

\begin{document}
\begin{frame}
\titlepage
Produced by \LaTeX\hspace{0.4em}Beamer
\end{frame}


\begin{frame}
\frametitle{题目介绍}
\begin{block}{第5题}
5.对平面中的任意向量定义内积为:$(x,y)=x_{1}y_{1}+x_{2}y_{2}$。

(1)试用该定义给出平面中向量的长度,以及任意两向量之间的夹角;

(2)试用内积定义证明平行四边形法则;

(3)试证明当范数满足平行四边形时,由它可以定义一个内积(先给出定义,再验证它满足内积的三条性质)。

注:平行四边形法则是指:平行四边形中两条对角线长度的平方和等于四个边长度的平方和,即$\left\|x+y\right\|^{2}+\left\|x-y\right\|^{2}=2(\left\|x\right\|^{2}+\left\|y\right\|^{2})$
\end{block}
\end{frame}

\begin{frame}[t]
\frametitle{题目分析}
\begin{block}{第(1)小题}
题中定义了向量内积,由内积的几何意义可知,向量的长度就是其在自身方向上的投影。两向量内积除以范数之积即得到向量夹角的余弦。
\end{block}
\begin{block}{第(2)小题}
已知内积的定义,由平行四边形两对角线与边长的关系即可得到平行四边形法则。
\end{block}
\begin{block}{第(3)小题}
满足平行四边形法则的范数是由内积诱导的,范数满足平行四边形法则时,可以定义一个内积,且满足内积的性质,即交换律、齐次性、分配率、非负性。
\end{block}
\end{frame}

\begin{frame}[t]
\frametitle{题目解答}
\begin{block}{第(1)小题}
设$\mathop{x}=(x_{1},x_{2}),\mathop{y}=(y_{1},y_{2})$为内积空间中的任意两向量

向量的长度为$\left\|\mathop{x}\right\|=\sqrt{(\mathop{x},\mathop{x})}=\sqrt{{x_{1}}^{2}+{{x}_{2}}^{2}}$

\hspace{6.3em}$\left\|\mathop{y}\right\|=\sqrt{(\mathop{y},\mathop{y})}=\sqrt{{y_{1}}^{2}+{{y}_{2}}^{2}}$

向量夹角的余弦值为

$\cos\widehat{(\mathop{x},\mathop{y})}=\cfrac{(x,y)}{\left\|\mathop{x}\right\|,\left\|\mathop{y}\right\|}=\cfrac{x_{1}y_{1}+x_{2}y_{2}}{\sqrt{{x_{1}}^{2}+{{x}_{2}}^{2}}\sqrt{{y_{1}}^{2}+{{y}_{2}}^{2}}}$

故向量之间的夹角为

$\widehat{(\mathop{x},\mathop{y})}=\arccos\cfrac{x_{1}y_{1}+x_{2}y_{2}}{\sqrt{{x_{1}}^{2}+{{x}_{2}}^{2}}\sqrt{{y_{1}}^{2}+{{y}_{2}}^{2}}}$
\end{block}
\end{frame}

\begin{frame}[c]
\frametitle{题目解答}
\begin{block}{第(2)小题}
由范数和内积的关系易得

$\begin{cases}
{\left\|x+y\right\|}^{2}=(x,x)+(x,y)+(y,x)+(y,y)&(i)\\
{\left\|x-y\right\|}^{2}=(x,x)-(x,y)-(y,x)+(y,y)&(ii)\\
\end{cases}$

(i)(ii)两式相加得到

${\left\|x+y\right\|}^{2}+{\left\|x-y\right\|}^{2}=2[(x,x)+(y,y)]=2(
{\left\|x\right\|}^{2}+
{\left\|y\right\|}^{2})$

\end{block}
\end{frame}

\begin{frame}[c]
\frametitle{题目解答}
\begin{block}{第(3)小题}
定义内积$(x,y)=\cfrac{{\left\|x+y\right\|}^{2}-{\left\|x-y\right\|}^{2}}{4}$

\begin{itemize}
\item 交换律

$\begin{aligned}(y,x)&=\cfrac{{\left\|y+x\right\|}^{2}-{\left\|y-x\right\|}^{2}}{4}\\&=\cfrac{{\left\|x+y\right\|}^{2}-{\left\|x-y\right\|}^{2}}{4}\\&=(x,y)\end{aligned}$
\end{itemize}
\end{block}
\end{frame}

\begin{frame}[c]
\frametitle{题目解答}
\begin{block}{第(3)小题}
\begin{itemize}
\item 齐次性

$\begin{aligned}
	(kx,y)&=\cfrac{{\left\|kx+y\right\|}^{2}-{\left\|kx-y\right\|}^{2}}{4}\\
	&=k\cfrac{{\left\|x+y\right\|}^{2}-{\left\|x-y\right\|}^{2}}{4}\\
	&=k(x,y)
\end{aligned}$
\end{itemize}	
\end{block}
\end{frame}

\begin{frame}[c]
\frametitle{题目解答}
\begin{block}{第(3)小题}
\begin{itemize}
\item 分配律

$\begin{aligned}
	(x+y,z)&=\cfrac{{\left\|x+y+z\right\|}^{2}-{\left\|x+y-z\right\|}^{2}}{4}\\
	&=\cfrac{{\left\|x+z\right\|}^{2}-{\left\|x-z\right\|}^{2}}{4}+\cfrac{{\left\|y+z\right\|}^{2}-{\left\|y-z\right\|}^{2}}{4}\\
	&=(x,z)+(y,z)
\end{aligned}$
\end{itemize}	
\end{block}
\end{frame}

\end{document}

